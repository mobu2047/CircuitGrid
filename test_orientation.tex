\documentclass[border=5mm]{standalone}
\usepackage{circuitikz}
\begin{document}

% Test all orientations for NPN, PNP, NMOS, PMOS
% Each component is placed at specific coordinates for comparison
% Grid: columns are orientations (Up, Right, Down, Left), rows are component types

\begin{circuitikz}[scale=0.8, transform shape]

% Row labels
\node at (-2, 9) {NPN};
\node at (-2, 6) {PNP};
\node at (-2, 3) {NMOS};
\node at (-2, 0) {PMOS};

% Column labels
\node at (0, 11) {Up};
\node at (4, 11) {Right};
\node at (8, 11) {Down};
\node at (12, 11) {Left};

% ========== NPN Row (y=9) ==========
% Up (0): rotate=-90, xscale=-1
\node[npn, rotate=-90, xscale=-1] (npn_up) at (0, 9) {};
\draw (npn_up.B) -- ++(0, 1) node[above] {B};
\draw (npn_up.C) -- ++(-1, 0) node[left] {C};
\draw (npn_up.E) -- ++(1, 0) node[right] {E};

% Right (1): xscale=-1
\node[npn, xscale=-1] (npn_right) at (4, 9) {};
\draw (npn_right.B) -- ++(1, 0) node[right] {B};
\draw (npn_right.C) -- ++(0, 1) node[above] {C};
\draw (npn_right.E) -- ++(0, -1) node[below] {E};

% Down (2): rotate=90, xscale=-1
\node[npn, rotate=90, xscale=-1] (npn_down) at (8, 9) {};
\draw (npn_down.B) -- ++(0, -1) node[below] {B};
\draw (npn_down.C) -- ++(1, 0) node[right] {C};
\draw (npn_down.E) -- ++(-1, 0) node[left] {E};

% Left (3): yscale=-1
\node[npn, yscale=-1] (npn_left) at (12, 9) {};
\draw (npn_left.B) -- ++(-1, 0) node[left] {B};
\draw (npn_left.C) -- ++(0, -1) node[below] {C};
\draw (npn_left.E) -- ++(0, 1) node[above] {E};

% ========== PNP Row (y=6) ==========
% Up (0): rotate=-90, xscale=-1
\node[pnp, rotate=-90, xscale=-1] (pnp_up) at (0, 6) {};
\draw (pnp_up.B) -- ++(0, 1) node[above] {B};
\draw (pnp_up.C) -- ++(-1, 0) node[left] {C};
\draw (pnp_up.E) -- ++(1, 0) node[right] {E};

% Right (1): xscale=-1
\node[pnp, xscale=-1] (pnp_right) at (4, 6) {};
\draw (pnp_right.B) -- ++(1, 0) node[right] {B};
\draw (pnp_right.C) -- ++(0, 1) node[above] {C};
\draw (pnp_right.E) -- ++(0, -1) node[below] {E};

% Down (2): rotate=90, xscale=-1
\node[pnp, rotate=90, xscale=-1] (pnp_down) at (8, 6) {};
\draw (pnp_down.B) -- ++(0, -1) node[below] {B};
\draw (pnp_down.C) -- ++(1, 0) node[right] {C};
\draw (pnp_down.E) -- ++(-1, 0) node[left] {E};

% Left (3): yscale=-1
\node[pnp, yscale=-1] (pnp_left) at (12, 6) {};
\draw (pnp_left.B) -- ++(-1, 0) node[left] {B};
\draw (pnp_left.C) -- ++(0, -1) node[below] {C};
\draw (pnp_left.E) -- ++(0, 1) node[above] {E};

% ========== NMOS Row (y=3) ==========
% Up (0): rotate=-90, xscale=-1
\node[nmos, rotate=-90, xscale=-1] (nmos_up) at (0, 3) {};
\draw (nmos_up.G) -- ++(0, 1) node[above] {G};
\draw (nmos_up.D) -- ++(-1, 0) node[left] {D};
\draw (nmos_up.S) -- ++(1, 0) node[right] {S};

% Right (1): xscale=-1
\node[nmos, xscale=-1] (nmos_right) at (4, 3) {};
\draw (nmos_right.G) -- ++(1, 0) node[right] {G};
\draw (nmos_right.D) -- ++(0, 1) node[above] {D};
\draw (nmos_right.S) -- ++(0, -1) node[below] {S};

% Down (2): rotate=90, xscale=-1
\node[nmos, rotate=90, xscale=-1] (nmos_down) at (8, 3) {};
\draw (nmos_down.G) -- ++(0, -1) node[below] {G};
\draw (nmos_down.D) -- ++(1, 0) node[right] {D};
\draw (nmos_down.S) -- ++(-1, 0) node[left] {S};

% Left (3): yscale=-1
\node[nmos, yscale=-1] (nmos_left) at (12, 3) {};
\draw (nmos_left.G) -- ++(-1, 0) node[left] {G};
\draw (nmos_left.D) -- ++(0, -1) node[below] {D};
\draw (nmos_left.S) -- ++(0, 1) node[above] {S};

% ========== PMOS Row (y=0) ==========
% Up (0): rotate=-90
\node[pmos, rotate=-90] (pmos_up) at (0, 0) {};
\draw (pmos_up.G) -- ++(0, 1) node[above] {G};
\draw (pmos_up.D) -- ++(-1, 0) node[left] {D};
\draw (pmos_up.S) -- ++(1, 0) node[right] {S};

% Right (1): xscale=-1, yscale=-1
\node[pmos, xscale=-1, yscale=-1] (pmos_right) at (4, 0) {};
\draw (pmos_right.G) -- ++(1, 0) node[right] {G};
\draw (pmos_right.D) -- ++(0, 1) node[above] {D};
\draw (pmos_right.S) -- ++(0, -1) node[below] {S};

% Down (2): rotate=90
\node[pmos, rotate=90] (pmos_down) at (8, 0) {};
\draw (pmos_down.G) -- ++(0, -1) node[below] {G};
\draw (pmos_down.D) -- ++(1, 0) node[right] {D};
\draw (pmos_down.S) -- ++(-1, 0) node[left] {S};

% Left (3): no transform
\node[pmos] (pmos_left) at (12, 0) {};
\draw (pmos_left.G) -- ++(-1, 0) node[left] {G};
\draw (pmos_left.D) -- ++(0, -1) node[below] {D};
\draw (pmos_left.S) -- ++(0, 1) node[above] {S};

\end{circuitikz}

\end{document}
