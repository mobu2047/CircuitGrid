\documentclass[border=5mm]{standalone}
\usepackage{circuitikz}
\begin{document}

% Test PNP vs NPN with the SAME transforms
% NPN works correctly, so let's verify PNP anchors

\begin{circuitikz}[scale=1.2, transform shape]

% Labels
\node at (0, 4) {\textbf{NPN}};
\node at (4, 4) {\textbf{PNP}};

% Row labels
\node at (-2, 3) {Up};
\node at (-2, 1) {Right};

% NPN Up: rotate=90, xscale=-1
\node[npn, rotate=90, xscale=-1] (npn_up) at (0, 3) {};
\node[above] at (npn_up.B) {B};
\node[left] at (npn_up.C) {C};
\node[right] at (npn_up.E) {E};

% PNP Up: rotate=90, xscale=-1 (same as NPN)
\node[pnp, rotate=90, xscale=-1] (pnp_up) at (4, 3) {};
\node[above] at (pnp_up.B) {B};
\node[left] at (pnp_up.C) {C};
\node[right] at (pnp_up.E) {E};

% NPN Right: xscale=-1
\node[npn, xscale=-1] (npn_right) at (0, 1) {};
\node[right] at (npn_right.B) {B};
\node[above] at (npn_right.C) {C};
\node[below] at (npn_right.E) {E};

% PNP Right: xscale=-1 (same as NPN)
\node[pnp, xscale=-1] (pnp_right) at (4, 1) {};
\node[right] at (pnp_right.B) {B};
\node[above] at (pnp_right.C) {C};
\node[below] at (pnp_right.E) {E};

\end{circuitikz}

\end{document}
