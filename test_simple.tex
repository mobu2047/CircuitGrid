\documentclass[border=5mm]{standalone}
\usepackage{circuitikz}
\begin{document}

% Simple test: Just show NMOS with different transforms, no connections
% This helps us understand what each transform does to the component

\begin{circuitikz}[scale=1.2, transform shape]

% Labels
\node at (0, 5) {\textbf{DEFAULT}};
\node at (3, 5) {\textbf{rotate=-90}};
\node at (6, 5) {\textbf{rotate=90}};
\node at (9, 5) {\textbf{xscale=-1}};
\node at (12, 5) {\textbf{yscale=-1}};

\node at (0, 2.5) {\textbf{r=-90,x=-1}};
\node at (3, 2.5) {\textbf{r=90,x=-1}};
\node at (6, 2.5) {\textbf{x=-1,y=-1}};

% Row 1: Basic transforms
\node[nmos] (n1) at (0, 4) {};
\node[above] at (n1.G) {G};
\node[above] at (n1.D) {D};
\node[below] at (n1.S) {S};

\node[nmos, rotate=-90] (n2) at (3, 4) {};
\node[above] at (n2.G) {G};
\node[right] at (n2.D) {D};
\node[left] at (n2.S) {S};

\node[nmos, rotate=90] (n3) at (6, 4) {};
\node[below] at (n3.G) {G};
\node[left] at (n3.D) {D};
\node[right] at (n3.S) {S};

\node[nmos, xscale=-1] (n4) at (9, 4) {};
\node[above] at (n4.G) {G};
\node[above] at (n4.D) {D};
\node[below] at (n4.S) {S};

\node[nmos, yscale=-1] (n5) at (12, 4) {};
\node[above] at (n5.G) {G};
\node[below] at (n5.D) {D};
\node[above] at (n5.S) {S};

% Row 2: Combined transforms
\node[nmos, rotate=-90, xscale=-1] (n6) at (0, 1) {};
\node[above] at (n6.G) {G};
\node[left] at (n6.D) {D};
\node[right] at (n6.S) {S};

\node[nmos, rotate=90, xscale=-1] (n7) at (3, 1) {};
\node[below] at (n7.G) {G};
\node[right] at (n7.D) {D};
\node[left] at (n7.S) {S};

\node[nmos, xscale=-1, yscale=-1] (n8) at (6, 1) {};
\node[above] at (n8.G) {G};
\node[below] at (n8.D) {D};
\node[above] at (n8.S) {S};

\end{circuitikz}

\end{document}
