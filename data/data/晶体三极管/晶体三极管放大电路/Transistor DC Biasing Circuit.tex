\documentclass[border=10pt]{standalone}
\usepackage{tikz}
\usepackage{circuitikz}
\tikzset{every node/.style={font=\large}}
\tikzset{every draw/.style={font=\large}}
\begin{document}
\begin{circuitikz}[line width=1pt]
\ctikzset{tripoles/en amp/input height=0.5};
\ctikzset{inductors/scale=1.2, inductor=american}
\ctikzset{tripoles/mos style/arrows}
\draw (3.0,24.0) to[short] (6.0,24.0);
\draw (3.0,24.0) to[short] (3.0,21.0);
\draw (3.0,21.0) to[generic, l=$RC$, ] (3.0,18.0);
\draw (9.0,21.0) to[generic, l=$RB1$, ] (12.0,21.0);
\draw (12.0,21.0) to[short] (15.0,21.0);
\draw (12.0,21.0) to[generic, l=$RB2$, ] (12.0,18.0);
\draw (15.0,21.0) to [short] (15.0,18.0);
\ctikzset{american};
\draw (15.0,21.0) to[rmeter, t, v=$VBB$] (15.0,18.0);
\ctikzset{european};
\draw (3.0,18.0) to [short] (6.0,18.0);
\ctikzset{american};
\draw (3.0,18.0) to[rmeter, t, v=$VCC$] (6.0,18.0);
\ctikzset{european};
\draw (6.0,18.0) to[short] (9.0,18.0);
\draw (6.0,18.0) to[short] (6.0,15.0);
\draw (9.0,18.0) to[short] (12.0,18.0);
\draw (12.0,18.0) to[short] (15.0,18.0);
% NPN/PNP Transistor Q_{T}
\node[npn, rotate=0, xscale=-1] (QT) at (6.0,21.0) {};
\node[right, rotate=0] at (6.5,21.0) {$T$};
\draw (QT.B) -- (9.0,21.0);
\draw (QT.C) |- (6.0,24.0);
\draw (QT.E) |- (6.0,18.0);
% GND GND_{0}
\draw (6.0,15.0) node[ground] {};
\node[circ] at (6.0,18.0) {};

\end{circuitikz}
\end{document}