\documentclass[border=10pt]{standalone}
\usepackage{tikz}
\usepackage{circuitikz}
\tikzset{every node/.style={font=\large}}
\tikzset{every draw/.style={font=\large}}
\begin{document}
\begin{center}
\begin{circuitikz}[line width=1pt]
\ctikzset{tripoles/en amp/input height=0.5};
\ctikzset{inductors/scale=1.2, inductor=american}
\draw (9.0,9.0) to[short] (6.0,9.0);
\draw (9.0,9.0) to[short] (9.0,6.0);
\draw (6.0,9.0) to[short] (3.0,9.0);
\draw (3.0,9.0) to[short] (0.0,9.0);
\draw (3.0,9.0) to[short] (3.0,6.0);
\draw (0.0,9.0) to[short] (0.0,6.0);
\draw (9.0,6.0) to[generic, l=$10 \mathrm{ k\Omega }$, ] (9.0,3.0);
\draw (3.0,6.0) to[short] (3.0,3.0);
\draw (0.0,6.0) to[generic, l=$18 \mathrm{ k\Omega }$, ] (0.0,3.0);
\draw (9.0,3.0) to [short] (9.0,0.0);
\ctikzset{american};
\draw (9.0,3.0) to[rmeter, t, v=$12 \mathrm{ V }$] (9.0,0.0);
\ctikzset{european};
\draw (3.0,3.0) to[generic, l=$50 \mathrm{ k\Omega }$, ] (3.0,0.0);
\draw (0.0,3.0) to[short] (0.0,0.0);
\draw (9.0,0.0) to[short] (6.0,0.0);
\draw (6.0,0.0) to[short] (3.0,0.0);
\draw (3.0,0.0) to[short] (0.0,0.0);
% NPN/PNP Transistor Q_{1}
\node[npn, rotate=-90] (Q1) at (6.0,6.0) {};
\node[below] at (6.0,5.5) {$Q_{1}$};
\draw (Q1.B) -- (9.0,6.0);
\draw (Q1.C) |- (3.0,9.0);
\draw (Q1.E) |- (3.0,3.0);

\end{circuitikz}
\end{center}
\end{document}